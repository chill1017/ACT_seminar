\section{9/17/2025}
I'd like to define a representaiton today, since we have all the prerequisite knowledge.
The crux of the problem is this: group multiplication is hard.
It's generally undecidable whether a given group element is even the identity.
But if we can translate a group's multiplication into something more concrete (like matrices!),
then we can learn a lot more about the group.

We'll begin by recalling that a homomorphism between groups $G$ and $H$ 
is a function $f:G\to H$ satisfying \[ f(\underbrace{ab}_{\in G}) = \underbrace{f(a)f(b)}_{\in H}. \]
It's important to remember which group the multiplication is happening in.

There's a cool way to visualize this property that I'll draw on the board.
I won't include it here becasue it's time-consuming to create\dots

\subsection{Representations}
Groups are meant to {\it act} on things, that is, to encode structure-preserving permutations.
We've seen examples of this already:
\begin{itemize}
    \item Permutation groups: all permutations of an abstract set $X$. The structure being preserved here is cardinality.
    \item Matrix groups: an invertible (bijective!) matrix is a permutation of $\C^n$.
\end{itemize}

A representation of a group is a sort of middle ground between these two.
That is, it translates some (maybe not all) of the structure of a group into a matrix group, which permutes a vector space.

\begin{definition}
    Let $G$ be a group.
    A (linear) {\bf representation} of $G$ is a homomorphism $\rho:G \to \GL_n(\C)$.
\end{definition}

\begin{remark}
    Soemtimes we will write $\rho(g)$ as $\rho_g$ and $[\rho(g)](v)$ as any one of $\rho_g v$, $g.v$, or even just $gv$.
\end{remark}

Every group has at least one representation.

\begin{example}
    Let $G$ be any group.
    Define $\rho:G\to \GL_1(\C)$ by $\rho(g) \coloneq [1]$ for every $g\in G$.
    Then $\rho$ defines a representation, since it's a homomorphism:
    \begin{align*}
        \rho(gh) & = [1] \\
        & = [1][1] \\
        & \rho(g) \rho(h)
    \end{align*}
    A similar construction gives a representation of $G$ on $\GL_n(\C)$ by sending $g\mapsto I_n$.
\end{example}

Here's an example that translates modular arithmetic into matrix multiplication.

\begin{example}
    Let $G = \Z_4 = \{ 0,1,2,3 \}$ with addition modulo 4 as the operation.
    Define $\rho:G\to \GL_1(\C)$ by $\rho(k) \coloneq [e^{2\pi i \frac{k}{4}}]$.
    Then $\rho$ defines a representation, since it's a homomorphism:
    \begin{align*}
        \rho(j+k) & = [ e^{2\pi i \frac{j+k}{4}} ] \\
        & = [ e^{2\pi i \frac{j}{4}} e^{2\pi i \frac{k}{4}} ] \\
        & = [ e^{2\pi i \frac{j}{4}} ][ e^{2\pi i \frac{k}{4}} ] \\
        & \rho(j) \rho(k)
    \end{align*}
\end{example}



\subsection{Inner automorphisms}
I planned to end this lecture talking about the geometric representation of $D_3$.
Instead, due to a question from Marek, we ended up talking about {\it inner automorphisms}.
The goal here is to morally prove that groups are made to act on things.
We'll do that by constructing, for {\it any} group $G$, a homomorphism $G\to S_G$ into 
the set of permutations of the group's elements.
This homomorphism won't be surjective; not every permutation of the set $G$ can be viewed
as a group element itself.
We'll actually be mapping into a subgroup of $S_G$ known as the automorphisms.
More specifically, inner automorphisms.

Henceforth, $G$ will be an arbitrary, but fixed, group.

\begin{definition}[Automorphism group]
    Let $G$ be any group, and define the set $\Aut(G)$ by
    \[
        \Aut(G) \coloneq \{ \phi: G\to G \mid \text{$\phi$ is an isomorphism}\}.
    \]
    A self-isomorphism $G\to G$ is called an {\bf automorphism}.
    Note that, in particular, an automorphism is a permutation.
\end{definition}

\begin{proposition}
    Fix a group $G$. The set $\Aut(G)$ of automorphisms of $G$ forms a group,
    with the operation being function composition and the identity element being
    the identity homomorphism $\id_G$.
\end{proposition}

\begin{proof}
    We have three things to check: 
    closure under the group operation, 
    existsence of an identity element,
    and invertibility.
    I'll outline the logic here.

    (Closure) This reduces to checking that the composition of homomorphisms is a homomorphism,
    and that the composition of bijective functions is bijective.

    (Identity) Check that the identity map $\id_G$ given by $\id_G(x)=x$ is a bijective homomorphism.

    (Inversion) Check that, for any automorphism $f$ of $G$, the inverse function $f^{-1}$ 
    (which only exists because $f$ is assumed bijective!!) is a bijective homomorphism.
\end{proof}

Now, for any $x\in G$, let $\sigma_x: G\to G$ be the function defined by
\[
    \sigma_x(g) \coloneq xgx^{-1}.
\]
A function of this form is called an {\bf inner automorphism}.
Its name assumes the following fact.

\begin{proposition}
    For any fixed $x\in G$, the function $\sigma_x:G\to G$ is an automorphism.
\end{proposition}
\begin{proof}
    This entails proving that $\sigma_x(gh) = \sigma_x(g)\sigma_x(h)$, and that $\sigma_x$ is a bijective function.
    To prove it's bijective, it suffices to find an inverse function.
    Naturally enough, one can show that the function $\sigma_{x^{-1}}$ is the inverse function to $\sigma_x$.
    The homomorphism condition is a neat exercise.
\end{proof}

Now the result mentioned earlier.

\begin{theorem}
    Define the function $\Sigma:G \to \Aut(G)$ by $\Sigma(x) \coloneq \sigma_x$.
    Then $\Sigma$ is an injective homomorphism.
    In particular, $\Sigma$ exactly defines an isomprphism
    \[
        G \xlongrightarrow{\cong} \Sigma(G).
    \]
    So $G$ is isomorphic to a subgroup of $S_G$.
\end{theorem}