\section*{10/8/2025}


\red{UNSTABLE}

We'll begin by finishing most of what we need from representation theory.
Then we'll define categories in general, and discuss how our big three examples fit in the framework.



For now we fix a group $G$.
We will see how representations of this fixed group interact.

\begin{definition}[Subrepresentation]
    Let $\rho:G\to\GL(V)$ be a representation of $G$.
    A subspace $U$ of $V$ is called {\bf $G$-invariant} if 
    \[
        \forall u\in U, g\in G, \quad \rho_g(u)\in U.
    \]
    Equivalently, we have, for every $g\in G$, $\rho_g|_U \in \GL(U)$.
\end{definition}
This is similar to how we informally said that a subgroup is a subset that's also a group (with the same operation),
and a subspace is a subset that's also a vector space (with the same operations).

To understand the definition of the maps in the category of $G$ representations, it first will be helpful to recall some shorthand.
We defined the following three things to be the same:
\begin{itemize}
    \item $\rho(g)(v)$
    \item $\rho_g(v)$
    \item $g\cdot v$ when $\rho$ is understood.
\end{itemize}


\begin{definition}[Homomorphism/intertwiner]
    Let $\rho:G\to\GL(V)$ and $\pi:G\to\GL(W)$ be two representations of the same gorup.
    A linear map $T:V\to W$ is called a {\bf homomorphism} or {\bf intertwiner}
    of representations if, for ever $g\in G$, we have $T(\rho_g(v)) = \pi_g(T(v))$.
    We express this by saying the following diagram {\it commutes}:
    \[
    \xymatrix@C=55pt{
        V \ar[r]^{T} \ar[d]^{\rho_g} & W \ar[d]^{\pi_g} \\
        V \ar[r]^{T} & W \\
    }
    \]
\end{definition}

Equivalently, we may express this equality by stating 
\begin{equation*}
    \forall g\in G, v\in V, \quad T(g\cdot v) = g\cdot T(v).
\end{equation*}
    We may express this in a more element-free way by merely writing
\begin{equation}\label{eq:rep-hom-condition}
    \forall g\in G, \quad T\circ \rho_g = \pi_g \circ T.
\end{equation}