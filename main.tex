\documentclass{article}
\usepackage{graphicx} % Required for inserting images
\usepackage{amssymb}
\usepackage{amsthm}
\usepackage{verbatim}
\usepackage{hyperref}
\usepackage{mathtools}
\usepackage{enumerate}

\usepackage{xy}
\xyoption{all}

%%%%%%%%%%%%% FORMATTING %%%%%%%%%%%%%

% Spacing
\usepackage{setspace}
\singlespacing


% Margins, etc
\usepackage{geometry} 

% Saving title space
\geometry{a4paper, margin=1in}
\usepackage{titling}
\setlength{\droptitle}{0em}

% Custom footer
\usepackage{fancyhdr}
\pagestyle{fancy}
\lhead{Caleb Hill}
\rhead{UNH Mathematics}
\lfoot{  }
\rfoot{  }

% % Bibliography
% \usepackage[
% backend=biber,
% style=numeric,
% % style=authoryear
% % style=bwl-FU
% ]{biblatex}
% \addbibresource{refs.bib}



%%%%%%%%%%%%%%%%% MARKUPS %%%%%%%%%%%%%%%%%%%%%%%
\usepackage[color=cyan]{todonotes}
\usepackage{color}
\newcommand{\red}[1]{{\color{red}{#1}}}
\newcommand{\blue}[1]{{\color{blue}{#1}}}
\newcommand{\cainsays}[1]{\blue{Cain says: #1}}

\usepackage{comment}



%%%%%%%%%%%%%%%%%% MATH %%%%%%%%%%%%%%%%%%%%%%

\newcommand{\onto}{\twoheadrightarrow}

\newcommand{\N}{\mathbb{N}}
\newcommand{\Z}{\mathbb{Z}}
\newcommand{\Q}{\mathbb{Q}}
\newcommand{\R}{\mathbb{R}}
\newcommand{\C}{\mathbb{C}}
\newcommand{\unit}{\mathbbm{1}} % depends on package bbm

\newcommand{\CC}{\mathcal{C}}
\newcommand{\DD}{\mathcal{D}}
\newcommand{\EE}{\mathcal{E}}
\newcommand{\FF}{\mathcal{F}}
\newcommand{\GG}{\mathcal{G}}
\newcommand{\MM}{\mathcal{M}}
\newcommand{\PP}{\mathcal{P}}
\newcommand{\Neg}{\mathcal{N}}

\newcommand{\tr}{tr}
\newcommand{\id}{id}
\newcommand{\ldag}{\langle}
\newcommand{\rdag}{\rangle^\dagger}
\newcommand{\ol}{\overline}

\newcommand{\dd}{\mathfrak{d}}
\newcommand{\ee}{\mathfrak{e}}
\renewcommand{\gg}{\mathfrak{g}}
\newcommand{\hh}{\mathfrak{h}}
\renewcommand{\sl}{\mathfrak{sl}}

\DeclareMathOperator{\ob}{ob}
\DeclareMathOperator{\im}{im}
\DeclareMathOperator{\Rep}{Rep}
\DeclareMathOperator{\Hom}{Hom}
\DeclareMathOperator{\Alg}{Alg}
\DeclareMathOperator{\GPA}{GPA}
\DeclareMathOperator{\End}{End}
\DeclareMathOperator{\Vecc}{Vec}
\DeclareMathOperator{\Sym}{Sym}
\DeclareMathOperator{\Kar}{Kar}
\DeclareMathOperator{\ev}{ev}
\DeclareMathOperator{\coev}{coev}
\DeclareMathOperator{\Res}{Res}
\DeclareMathOperator{\Ab}{Ab}


\renewcommand{\phi}{\varphi}
\let\oldepsilon\epsilon\let\epsilon\varepsilon\let\smallepsilon\oldepsilon %this makes \epsilon produce the curly epsilon, and \smallepsilon the smaller epsilon

\newcommand{\blank}{\rule{0.2cm}{0.15mm}}


%%%%%%%%%%%%%%%%%% THEOREMS %%%%%%%%%%%%%%%%%%%%%%
\theoremstyle{plain}
    \newtheorem{definition}{Definition}
    \newtheorem{example}{Example}
    \newtheorem{proposition}{Proposition}
    \newtheorem{theorem}{Theorem}
    \newtheorem{lemma}{Lemma}
    \newtheorem{corollary}{Corollary}
    \newtheorem{conjecture}{Conjecture}
    \newtheorem{remark}{Remark}



% skein diagrams
\newcommand{\skein}[2]{\raisebox{-.4\height}{ \includegraphics[scale = #2]{figs/#1.png}}}


\title{Applied Category Theory}
\author{Caleb Hill}
\date{Fall 2025}

\begin{document}
\maketitle


\section*{9/3/2025}

\subsection*{Why?}
Why should you care about studying the coming content and applying it to your field?

Physicists:
\begin{itemize}
    \item The particles in the standard model {\it are} irreducible representations. 
    So rep theory is crucial to you.
    \item Monoidal categories give a good framework for understanding QM.
\end{itemize}
Computer scientists:
\begin{itemize}
    \item It gives a framework for the Curry-Howard correspondence (proofs are programs).
\end{itemize}
Me:
\begin{itemize}
    \item Began as a study of ``analogies'' and turned into a study of nifty algebraic gadgets.
    \item It's written in a {\it wild} language, and learning languages is fun.
\end{itemize}


\subsection*{Plan}
The {\bf first goal} is to define monoidal categories with some context. 
The {\bf second goal} is to describe a ``skeletal'' category defined by diagrammatics.
To accomplish the first goal, we will study things including:
\begin{itemize}
    \item Algebraic objects (groups, vector spaces, ...) and maps between them 
    \item Subobjects, images, combining objects ($\times$, $\otimes$, $\oplus$, ...) 
    \item Categories ($Grp$, $Set$, $PoSet$, $\N$, $\Vecc$, ...)
\end{itemize}
I'd like to have as little fat on this as necessary.
That is, not get sidetracked studying, for instance, too much of the internal structure of these objects.
I want to give many examples and try to build intuition.
For the second goal we'll study things including:
\begin{itemize}
    \item Representations and maps ($T(gv) = gT(v)$)
    \item $\Rep(D_3)$ in detail
\end{itemize}
Up to this point I have a strong vision of where we're going. 
After this we can go where the interest steers us.

This plan is incomplete and non-exhaustive.




\subsection*{Groups}
This is the best onramp to categories I know of, so bear with me through some basics.
\begin{definition}
    A {\bf group} is a triple \[ (G,\mu,e)\] where $G$ is a set, 
    $\mu:G\times G\to G$ is a function, and $e\in G$, such that 
    \begin{equation*}\tag{Associativity}
        \forall a,b,c\in G, \quad \mu(\mu(a,b),c) = \mu(a,\mu(b,c))
    \end{equation*}
    \begin{equation*}\tag{Identity}
        \forall a\in G, \quad \mu(a,e) = \mu(e,a) = a
    \end{equation*}
    \begin{equation*}\tag{Inverse}
        \forall a\in G, \exists b\in G, \quad\text{such that}\quad \mu(a,b)=\mu(b,a)=e
    \end{equation*}
\end{definition}


\end{document}
