\section{9/24/2025}
\red{UNSTABLE}


\subsection{Geometric example}
These examples are both one-dimensional, so they don't really show the true flavor of representation theory. 
Here's an example that is more complicated, which we'll linger on for a while.
It's similar to the matrix group example, up to a change of basis.

Let $D_3$ be the group with the following six elements:
\[
    \{ e, r, r^2, s, sr, sr^2 \}
\]
where we can freely multiply any elements using the reduction rules
\[
    r^3 = e, \quad s^2 = e, \quad rs = sr^2.
\]
We could alternatively phrase this by saying $D_3 = \langle r,s \mid r^3 = e, s^2 = e, rs = sr^2 \rangle$.
We'll define a representation $\rho: D_3 \to \GL_2(\C)$ by declaring
\[
    \rho(s) \coloneq \begin{bmatrix}
        0 & 1 \\
        1 & 0 \\    
    \end{bmatrix}
    \quad \text{and} \quad
    \rho(r) \coloneq \begin{bmatrix}
        e^{2\pi i/3} & 0\\
        0 & e^{-2\pi i/3} \\
    \end{bmatrix}
\]
and claiming that with these assignments, the images of the rest of the elements of $D_3$ are fully determined.
To understand this claim, we should do an example or two.
We'll be as loose as you're comfortable with, and try to argue convincingly, albeit informally,
that $\rho$ translates the multiplication of $D_3$ into the multiplication in $\GL_2(\C)$.

First consider the properties $s^2=e$ and $r^3 = e$ separately in $D_3$.
The image of $s$ under $\rho$ shares a corresponding property:
\[
    \begin{bmatrix}
        0 & 1 \\
        1 & 0 \\    
    \end{bmatrix} \begin{bmatrix}
        0 & 1 \\
        1 & 0 \\    
    \end{bmatrix} = \begin{bmatrix}
        1 & 0 \\
        0 & 1 \\    
    \end{bmatrix}
\]
as does the image of $r$ under $\rho$:
\[
    \begin{bmatrix}
        e^{2\pi i/3} & 0\\
        0 & e^{-2\pi i/3} \\
    \end{bmatrix} \begin{bmatrix}
        e^{2\pi i/3} & 0\\
        0 & e^{-2\pi i/3} \\
    \end{bmatrix} \begin{bmatrix}
        e^{2\pi i/3} & 0\\
        0 & e^{-2\pi i/3} \\
    \end{bmatrix} = \begin{bmatrix}
        1 & 0 \\
        0 & 1 \\    
    \end{bmatrix}
\]