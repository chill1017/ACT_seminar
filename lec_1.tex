\section{9/3/2025}

\subsection{Why?}
Why should you care about studying the coming content and applying it to your field?

Physicists:
\begin{itemize}
    \item The particles in the standard model {\it are} irreducible representations. 
    So rep theory is crucial to you.
    \item Monoidal categories give a good framework for understanding QM.
\end{itemize}
Computer scientists:
\begin{itemize}
    \item It gives a framework for the Curry-Howard correspondence (proofs are programs).
\end{itemize}
Me:
\begin{itemize}
    \item Began as a study of ``analogies'' and turned into a study of nifty algebraic gadgets.
    \item It's written in a {\it wild} language, and learning languages is fun.
\end{itemize}
As a great motivation, see \url{https://arxiv.org/abs/0903.0340}


\subsection{Plan}
The {\bf first goal} is to define monoidal categories with some context. 
The {\bf second goal} is to describe a ``skeletal'' category defined by diagrammatics.
To accomplish the first goal, we will study things including:
\begin{itemize}
    \item Algebraic objects (groups, vector spaces, ...) and maps between them 
    \item Subobjects, images, combining objects ($\times$, $\otimes$, $\oplus$, ...) 
    \item Categories ($Grp$, $Set$, $PoSet$, $\N$, $\Vecc$, ...)
\end{itemize}
I'd like to have as little fat on this as necessary.
That is, not get sidetracked studying, for instance, too much of the internal structure of these objects.
I want to give many examples and try to build intuition.
For the second goal we'll study things including:
\begin{itemize}
    \item Representations and maps ($T(gv) = gT(v)$)
    \item $\Rep(D_3)$ in detail
\end{itemize}
Up to this point I have a strong vision of where we're going. 
After this we can go where the interest steers us.

This plan is incomplete and non-exhaustive.




\subsection{Groups}
This is the best onramp to categories I know of, so bear with me through some basics.
\begin{definition}
    A {\bf group} is a triple \[ (G,\mu,e)\] where $G$ is a set, 
    $\mu:G\times G\to G$ is a function, and $e\in G$, such that 
    \begin{equation*}\tag{Associativity}
        \forall a,b,c\in G, \quad \mu(\mu(a,b),c) = \mu(a,\mu(b,c))
    \end{equation*}\label{axiom:group-assoc}
    \begin{equation*}\tag{Identity}
        \forall a\in G, \quad \mu(a,e) = \mu(e,a) = a
    \end{equation*}\label{axiom:group-id}
    \begin{equation*}\tag{Inverse}
        \forall a\in G, \exists b\in G, \quad\text{such that}\quad \mu(a,b)=\mu(b,a)=e
    \end{equation*}\label{axiom:group-inv}
    We often call the element $b$ from \ref{axiom:group-inv} by $a^{-1}$.
    We also often use the following shorthands:
    \begin{itemize}
        \item $\mu(a,b) = a\cdot b = a\star b = ab$
        \item $\underbrace{a\cdot a\cdots a}_{\text{$n$ copies}} = a^n$
    \end{itemize}
\end{definition}

\begin{exercise}
    Translate the three axioms above into the $ab$ notation.
\end{exercise}

\begin{exercise}
    Prove the identity element in a group is unique. 
    That is, if $e$ and $e'$ both satisfy Axiom~\ref{axiom:group-id},
    show that $e'=e$.
\end{exercise}

Now some examples. 
As an excercise, prove that each of the following is a group. 
The notation $\coloneq$ reads as ``is defined to be.''

\begin{example}[General linear group]
    $(G,\mu,e)$, where
    \begin{itemize}
        \item $G = \GL_2(\C) \coloneq \{ \text{invertible $2\times 2$ matrices with entries in $\C$}\}$
        \item $\mu(A,B) \coloneq AB$
        \item $e=I_2 = \begin{bmatrix} 1&0\\0&1\\ \end{bmatrix}$
    \end{itemize}
\end{example}

\begin{example}[General linear group]
    $(G,\mu,e)$, where
    \begin{itemize}
        \item $G = \GL_n(\C) \coloneq \{ \text{invertible $n\times n$ matrices with entries in $\C$}\}$
        \item $\mu(A,B) \coloneq AB$
        \item $e = I_n$ (the $n\times n$ identity matrix)
    \end{itemize}
\end{example}

\begin{example}[Integers]
    $(G,\mu,e)$, where
    \begin{itemize}
        \item $G = \Z$
        \item $\mu(a,b) \coloneq a+b$
        \item $e = 0$
    \end{itemize}
\end{example}

\begin{example}[\red{Not a group! Why?}]
    $(G,\mu,e)$, where
    \begin{itemize}
        \item $G = \Z$
        \item $\mu(a,b) \coloneq a\times b$
        \item $e = 1$
    \end{itemize}
\end{example}

\begin{example}[Braid group]
    $B_n \coloneq (G,\mu,e)$, where
    \begin{itemize}
        \item $G = \text{$n$-strand braid diagrams (up to isotopy/wiggling)}$
        \item $\mu = \text{vertical concatenation}$
        \item $e = \text{$n$ unbraided strands}$
    \end{itemize}
    Steve pointed out that when $n=2$, $B_n$ is {\it isomorphic} to $\Z$.
    We'll get to that in the next lecture I hope.
\end{example}

\subsection{Things that came up}
\begin{itemize}
    \item Generators and relations presentations
    \item Free group/group of words
    \item Symmetric group/permutation groups
    \item The natural numbers game: \\ 
    \url{https://www.ma.imperial.ac.uk/~buzzard/xena/natural_number_game/index2.html}
    \item Peano arithmetic: \\ 
    \url{https://en.wikipedia.org/wiki/Peano_axioms}
\end{itemize}