\section{9/10/2025}

\subsection{More groups}
Recall that, loosely, a group is a set endowed with a binary operation (multiplication), 
with some associativity, identity, and inversion constraints.
Henceforth we will refer to a group $(G,\mu,e)$ almost exclusively as $G$.
Usually, $\mu$ and $e$ will be understood from context.
For some of the following examples we might also write 
$\mu_G$ to refer specifically to multiplication in $G$.




\begin{example}[Nonzero field elements]
    The set $\C^\times = \C-\{0\}$ is a group under multiplication.
    Associativity is known.
    Its identity element is $1\in\C$.
    This is true for any field, i.e. $\R^\times$, $\Q^\times$, and $\FF_{p^n}^\times$
    are all groups.
\end{example}



\begin{example}[Symmetric groups]
    Let $X$ be a set. Then $(G,\mu,e)$ is a group, where
    \begin{itemize}
        \item $G = \{ \text{$\sigma:X\to X \mid \text{$\sigma$ is bijective}$ } \}$ %bad tex, I know
        \item $\mu(\sigma_1,\sigma_2) \coloneq \sigma_1\circ\sigma_2$
        \item $e = \id_X$, defined by $\forall x\in X$, $\id_X(x) = x$
    \end{itemize}
\end{example}

\begin{example}
    If $X=\{1,\dots,n\}$, then we denote the group $S_X$ by $S_n$.
    If $\sigma\in S_n$ then we often denote $\sigma$ as 
    \[
        \sigma = \begin{pmatrix}
            1 & 2 & \cdots & n \\
            \sigma(1) & \sigma(2) & \cdots & \sigma(n)
        \end{pmatrix}
    \]
\end{example}




\begin{example}[Dihedral group]
    Let $G$ consist of the following six matrices:
    % \left\{ \begin{align*}
    %     G & =   
    %         \begin{bmatrix}
    %             1 & 0\\ 
    %             0 & 1
    %         \end{bmatrix}, & \begin{bmatrix}
    %             \cos\frac{2\pi}{3} & -\sin\frac{2\pi}{3}\\ 
    %             \sin\frac{2\pi}{3} & \cos\frac{2\pi}{3}
    %         \end{bmatrix}, & \begin{bmatrix}
    %             \cos\frac{4\pi}{3} & -\sin\frac{4\pi}{3}\\ 
    %             \sin\frac{4\pi}{3} & \cos\frac{4\pi}{3}
    %         \end{bmatrix},\\
    %         & \begin{bmatrix}
    %             1 & 0\\ 
    %             0 & -1
    %         \end{bmatrix}, & \begin{bmatrix}
    %             \cos\frac{2\pi}{3} & \sin\frac{2\pi}{3}\\ 
    %             \sin\frac{2\pi}{3} & -\cos\frac{2\pi}{3}
    %         \end{bmatrix}, & \begin{bmatrix} 
    %             \cos\frac{4\pi}{3} & \sin\frac{4\pi}{3}\\ 
    %             \sin\frac{4\pi}{3} & -\cos\frac{4\pi}{3}
    %         \end{bmatrix}
    %     \end{align*}  
    % \right\}
\end{example}

These matrices permute the points
\[
    \left\{ (1,0), 
    ( -\frac{1}{2},\frac{\sqrt{3}}{2} ), 
    ( -\frac{1}{2},-\frac{\sqrt{3}}{2} ) \right\}
\]
So we can see that it's ``the same as'' a certain permutation group.
There are really 3 things at play in this last example:
\begin{itemize}
    \item Dihedral groups: 3 wasn't special. 
    We could divide by any positive $n$ and get a group of $2n$ matrices.
    \item Generators and relations: We could equally express this group as
    \[ \langle r,s \mid r^3=e, s^2=e, rs=sr^2 \rangle.\]
    In fact, that's usually how dihedral groups are presented.
    \item Isomorphism: That group of matrices ``is'' a permutation group.
\end{itemize}





\subsection{Homomorphisms}
We'll start by definine how two groups can be similar, or the same.
\begin{definition}
    Let $G$ and $H$ be two groups.
    Let $f:G\to H$ be a function.
    \begin{itemize}
        \item We call $f$ a {\bf homomorphism} if 
            \[ \forall a,b\in G, f(ab)=f(a)f(b)\]
        \item We call $f$ an {\bf isomorphism} if it is a bijective homomorphism
        \item If $f$ is a homomorphism, the {\bf kernel} of $f$ is the set
            \[ \ker(f)\coloneq \{x\in G\mid f(x)=e_H\}\]
        \item If $f$ is a homomorphism, the {\bf image} of $f$ is the set
            \[ f(G) \coloneq \{f(X) \mid x\in G\} \subseteq H\]
    \end{itemize}
\end{definition}

Here are many examples. 
It would be useful to prove those you don't see immediately.
Well, it would probably be good to prove all of them...

\begin{example}[Modular arithmetic]
    $f: \Z \to \Z_n$, given by $f(x) \coloneq x\pmod n$
\end{example}

\begin{example}[Multiplication]
    $f: \Z \to \Z$, given by $f(x) \coloneq 4x$.
    What is special about 4 here? Anything?
\end{example}

\begin{example}[Trivial]
    Let $G$ and $H$ be any two groups.
    Define $f: G \to H$ by $f(x) = e$.
\end{example}

\begin{example}[Identity]
    Let $G$ be any group.
    Define $f:G\to G$ by $f(x) \coloneq x$.
\end{example}

\begin{example}[Symmetric group inclusion]
    Define $f:S_n \to S_{n+1}$ by declaring 
    \[
        f(\sigma) \coloneq \begin{pmatrix}
            1 & 2 & \cdots & n & n+1 \\
            \sigma(1) & \sigma(2) & \cdots & \sigma(n) & n+1 \\
        \end{pmatrix}
    \]
    In particular, since this homomorphism is injective, this means we can think of $S_n$ as
    ``sitting inside of'' $S_{n+1}$.
    This actually holds more generally.
    If a homomorphism $f:G\to H$ is injective, then there is an isomorphic copy of $G$ inside of $H$,
    in the form of the image $f(G)$.
    We'll define this more precisely in the next subsection.
\end{example}

\begin{example}[Linear maps]
    Let $V$ and $W$ be vector spaces over $\C$.
    The definition of vector spaces says that $(V,+)$ and $(W,+)$ are, in particular, abelian groups.
    Let $T:V\to W$ be a linear map.
    The condition
    \[
        T(x+y)=T(x) + T(y)
    \]
    implies that $T$ is a homomorphism of (abelian) groups.
\end{example}

\begin{example}[All linear maps]
    $f: \Z \to \Z_n$, given by $f(x) \coloneq x\pmod n$
\end{example}

Here's a fact that we'll have uses for.
\begin{exercise}
    Let $f:G\to H$ be a homomorphim of groups.
    For every $a\in G$, the inverse of $f(a)$ is $f(a^{-1})$.
    In equation form, that's
    \[ [f(a)]^{-1} = f(a^{-1}).\]
\end{exercise}








\subsection{Subgroups}
As we saw above, $S_n$ ``sits inside of'' $S_{n+1}$.
Here's how we say that precisely.

\begin{definition}
    Let $G$ be a group.
    A nonempty subset $A\subseteq G$ is called a subgroup of $G$ if 
    \begin{equation*}\tag{Closure}
        \forall a,b\in A, \quad ab\in A
    \end{equation*}

    \begin{equation*}\tag{Inversion}
        \forall a\in A, \quad a^{-1}\in A
    \end{equation*}
\end{definition}

The first consequence of this definition is that if $A$ is a subgroup of $G$, then $e\in A$.
Why? Well, take any element $a\in A$.
Its inverse, $a^{-1}$ also is in $A$ by definition.
Their product also must be in $A$.
But their product is $e$.

Here's a fact that will be of use.
\begin{proposition}
    Let $f:G\to H$ be a homomorphism.
    Then
    \begin{enumerate}
        \item The kernel $\ker(f)$ is a subgroup of $G$.
        \item The image $f(G)$ is a subgroup of $H$.
    \end{enumerate}
\end{proposition}

Let's see some examples of subgroups in action.

\begin{example}[Dihedral groups]
    Let $n\geq1$ be a positive integer.
    The set of $2n$ matrices of the forms
    \[
        \begin{bmatrix}
            \cos2\pi\frac{k}{3} & -\sin2\pi\frac{k}{3}\\ 
            \sin2\pi\frac{k}{3} & \cos2\pi\frac{k}{3}
        \end{bmatrix} \quad \text{and} \quad
        \begin{bmatrix}
            \cos2\pi\frac{k}{3} & \sin2\pi\frac{k}{3}\\ 
            \sin2\pi\frac{k}{3} & -\cos2\pi\frac{k}{3}
        \end{bmatrix}
    \]
    for $k=0,1,\dots,n-1$ is a subgroup of $\GL_2(\C)$.
\end{example}

\begin{example}[Special linear group]
    The set $SL_N(\C) \coloneq \{M \in \GL_N(\C) \mid \det(M)=1 \}$ 
    is a subgroup of the general linear gorup $\GL_N(\C)$.
\end{example}