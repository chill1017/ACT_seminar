\section*{10/15/2025}

\red{UNSTABLE}

We'll begin by finishing most of what we need from representation theory.
Then I'll try and actually sketch the so-called ``graphical correspondence.''

To understand the definition of the maps in the category of $G$ representations, it first will be helpful to recall some shorthand.
We defined the following three things to be the same:
\begin{itemize}
    \item $\rho(g)(v)$
    \item $\rho_g(v)$
    \item $g\cdot v$ or $gv$ when $\rho$ is understood.
\end{itemize}


\begin{definition}[Homomorphism/intertwiner]
    Let $\rho:G\to\GL(V)$ and $\pi:G\to\GL(W)$ be two representations of the same gorup.
    A linear map $T:V\to W$ is called a {\bf homomorphism} 
    of representations if, for every $g\in G$, we have $T(\rho_g(v)) = \pi_g(T(v))$.
    We express this by saying the following diagram {\it ``commutes''}:
    \[
    \xymatrix@C=55pt{
        V \ar[r]^{T} \ar[d]^{\rho_g} & W \ar[d]^{\pi_g} \\
        V \ar[r]^{T} & W \\
    }
    \]
\end{definition}

Equivalently, we may express this equality by stating 
\begin{equation*}
    \forall g\in G, v\in V, \quad T(g\cdot v) = g\cdot T(v).
\end{equation*}
    We may express this in a more element-free way by merely writing
\begin{equation}\label{eq:rep-hom-condition}
    \forall g\in G, \quad T\circ \rho_g = \pi_g \circ T.
\end{equation}

As promised, the representations of a group form a category.
It's actually a (rigid, pivotal, etc...) tensor category, but we'll get to some of those adjectives later.
\begin{example}
    Let $G$ be a group, and let $\Rep(G)$ denote the class of finite-dimensional representations of $G$.
    \begin{itemize}
        \item $ob(\Rep(G)) = \{ (V,\rho) \mid \text{$\rho:G\to\GL(V)$ is a representation}\}$
        \item $\Hom_{\Rep(G)}(V\to W) = \{ T:V\to W \mid \text{$T$ is an intertwiner} \}$
    \end{itemize}
\end{example}


For now we fix a finite group $G$.
We will see how representations of this fixed group interact.

\begin{definition}[Irreducibility]
    Let $\rho:G\to\GL(V)$ be a representation of $G$.
    A subspace $U$ of $V$ is called a {\bf subrepresentation} if 
    \[
        \forall u\in U, g\in G, \quad \rho_g(u)\in U.
    \]
    Equivalently, we have, for every $g\in G$, $\rho_g|_U \in \GL(U)$.

    A representation with no \red{nontrivial} subrepresentations is called {\bf irreducible}. 
\end{definition}

Irreducible representations of a group are the atomic objects in a representation category.
Every (finite-dimensional) representation of $G$ is a {\it direct sum} of irreducible representations.

Here's a lemma that we'll use to count multiplicities of subrepresentaitons.
\begin{lemma}[Schur's lemma]
    Let $\rho:G\to\GL(V)$ and $\pi:G\to\GL(W)$ be irreducible representations.
    Suppose $T:V\to W$ is a homomorphism of representations.
    Then either $T=0$ or $T$ is an isomorphism.
\end{lemma}

\begin{proof}
    It suffices to prove that $\ker(T)$ is a subrepresentation of $V$, 
    and $T(V)$ is a subrepresentation of $W$.
    If $T$ is nonzero, the former must be $\{\vec{0}\}$.
    Now, it can't be that the image $T(V)$ is anything but all of $W$.
    Hence $T$ has trivial kernel, and is surjective; an isomorphism.
\end{proof}

Now here's the form we'll use this lemma in.
\begin{corollary}
    Let $\rho:G\to V$ be an irreducible representation, and let $\pi:G\to X$ be any representation.
    If there is a nonzero homomorphism of representations
    \[
        T:V\to X
    \]
    then $X$ has a subrepresentation isomorphic to $V$.
\end{corollary}

In this case, we say ``$X$ contains a copy of $V$,''
and will sometimes write $V\leq X$.

The effect of this lemma is to allow us to find ``decompositions'' of representations by findind maps from irreducibles.
Let's set the stage.

Recall how we defined the dihedral group: 
\[
    D_3 = \langle r,s \mid r^3=s^2=e, sr=r^2s \rangle.
\]
Now we'll define three representations by saying what they do to the generators $r$ and $s$.

\begin{itemize}
    \item Define $\pi_t:G\to\GL(\C_t)$ by
    \[
    \pi_t(r) \coloneq 1 \bigand 
    \pi_t(s) \coloneq 1.
    \]

    \item Define $\pi_s:G\to\GL(\C_s)$ by
    \[
    \pi_s(r) \coloneq 1 \bigand 
    \pi_s(s) \coloneq -1.
    \]
    
    \item Define $\rho:G\to\GL(\C^2)$ by
    \[
    \rho(r) \coloneq \begin{bmatrix}
        e^{2\pi i/3} & 0\\
        0 & e^{-2\pi i/3} \\
    \end{bmatrix} \bigand 
    \rho(s) \coloneq \begin{bmatrix}
        0 & 1 \\
        1 & 0 \\
    \end{bmatrix}.
    \]

    \item Define $\rho\otimes\rho:G\to\GL(\C^2\otimes\C^2)$ by
    \[
    \rho\otimes\rho(r) \coloneq \begin{bmatrix}
        e^{2\pi i/3} \rho(r) & 0\\
        0 & e^{-2\pi i/3} \rho(r) \\
    \end{bmatrix} = \begin{bmatrix}
    e^{4\pi i/3} & 0 & 0 & 0 \\
    0 & 1 & 0 & 0 \\
    0 & 0 & 1 & 0 \\
    0 & 0 & 0 & e^{2\pi i/3} \\
    \end{bmatrix}
    \]
    and 
    \[
    \rho\otimes\rho(s) \coloneq \begin{bmatrix}
        0 & 1 \rho(s) \\
        1\rho(s) & 0 \\
    \end{bmatrix} = \begin{bmatrix}
    0 & 0 & 0 & 1 \\
    0 & 0 & 1 & 0 \\
    0 & 1 & 0 & 0 \\
    1 & 0 & 0 & 0 \\
    \end{bmatrix}.
    \]
\end{itemize}

Our goal here is to find morphisms 
\[
    \C_t \to\C^2\otimes\C^2,\quad \C_s\to \C^2\otimes\C^2,\bigand \C^2\to \C^2\otimes\C^2,
\]
thereby showing that $\C^2\otimes\C^2 = \C_t \oplus \C_s \oplus \C^2$.