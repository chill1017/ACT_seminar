\section*{date TBD}

\red{UNSTABLE}

I want to take a step back and examine some parts of the definition of a representation.
We defined a (complex, linear) representation as a homomorphism $G\to \GL_n(\C)$.
I did this to be a bit more concrete and avoid discussing vector spaces in their own right.
But I now realize that was a tactical mistake, and we should discuss vector spaces.

\begin{definition}
    A (complex) {\bf vector space} is a tuple $(V, s)$ where $V$ is a set, 
    and $s:\C \times V \to V$ is a function,
    satisfying the following properties:
    \begin{equation*}\tag{Addition}
        \forall u,v\in V, \quad u+v\in V
    \end{equation*}

    \begin{equation*}\tag{Zero}
        \exists \vec{0}\in V; \forall v\in V \quad \vec{0}+v = v+\vec{0} = v
    \end{equation*}

    \begin{equation*}\tag{Negative}
        \forall u\in V, \exists v\in V; \quad u+v = v+u = \vec{0}
    \end{equation*}

    \begin{equation*}\tag{Left distribute}
        \red{ \forall u,v\in V, \quad u+v\in V }
    \end{equation*}

    \begin{equation*}\tag{Right distribute}
        \red{ \forall u,v\in V, \quad u+v\in V }
    \end{equation*}
    where we have shortened $s(\lambda,v)$ to $\lambda v$.
\end{definition}

Some expected properties follow immediately from this definition.

\begin{exercise}
    Prove the following.
    \begin{itemize}
        \item $v + (-1)v = \vec{0}$
        \item $0 v=\vec{0}$
    \end{itemize}
\end{exercise}

There are many familiar examples.

\begin{example}[Zero]
    $\{ \vec{0} \}$
\end{example}


\begin{example}[$\C$]
    
\end{example}


\begin{example}[Tuples]
    $\C^n$
\end{example}


\begin{example}[Matrices]
    $M_{m\times n}(\C)$
\end{example}


\begin{example}[Polynomials]
    $\C[x]$
\end{example}


\begin{example}[Functions]
    $\C^{X}$
\end{example}




Sometimes a vector space sits inside another vector space.

\begin{definition}[Subspace]
    Let $V$ be a vector space, and let $U$ be a subset of $V$.
    Call $U$ a {\bf vector subspace} of $V$ if
    \begin{equation*}\tag{Addition}
        \forall u,v\in V, \quad u+v\in V
    \end{equation*}

    \begin{equation*}\tag{Scalar multiplication}
        \forall u\in U, \lambda\in\C, \quad \lambda u\in U
    \end{equation*}
\end{definition}

\begin{exercise}
    \begin{itemize}
        \item $\vec{0} \in U$
        \item $-u\in U$ 
    \end{itemize}
\end{exercise}

What kinds of functions do we care about for vector spaces?
Well, we have two sorts of structure now, so we want a function to respect both.

\begin{definition}[Linear transformation]
    Let $V$ and $W$ be vector spaces.
    A function $T:V\to W$ is called a {\bf linear transformation} if
    \[
        T(u+v) = T(u) + T(w), \quad \text{and} \quad T(\lambda v) = \lambda T(v).
    \]
\end{definition}