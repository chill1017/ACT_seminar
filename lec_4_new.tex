\section*{10/1/2025}


I want to take a step back and examine some parts of the definition of a representation.
We defined a (complex, linear) representation as a homomorphism $G\to \GL_n(\C)$.
I did this to be a bit more concrete and avoid discussing vector spaces in their own right.
But I now realize that was a tactical mistake, and we should discuss vector spaces as such.
The reason is that in order to build a category theory that doesn't seem conjured,
we should have a few more examples of objects, subobjects, and maps.
Discussing vector spaces in their own right does this twofold: once for vector spaces, 
and then we'll build on that for representations.

\begin{definition}
    A (complex) {\bf vector space} is a tuple $(V, +, s)$ where $V$ is a set, 
    and $s:\C \times V \to V$ is a function,
    satisfying the following properties:
    \begin{equation*}\tag{Commutative}
        \forall u,v\in V, \quad u+v = v+u\in V
    \end{equation*}

    \begin{equation*}\tag{Associative}
        \forall u,v,w\in V, \quad (u+v)+w = u+(v+w)\in V
    \end{equation*}

    \begin{equation*}\tag{Zero}
        \exists \vec{0}\in V; \forall v\in V \quad \vec{0}+v = v+\vec{0} = v
    \end{equation*}

    \begin{equation*}\tag{Negative}
        \forall u\in V, \exists v\in V; \quad u+v = v+u = \vec{0}
    \end{equation*}

    \begin{equation*}\tag{Scalar assoc.}
        \forall \alpha,\beta\in \C, v\in V \quad \alpha\beta(v) = \alpha(\beta v)
    \end{equation*}

    \begin{equation*}\tag{Left distribute}
        \forall \alpha,\beta\in \C, v\in V \quad (\alpha+\beta)v = \alpha v + \beta v 
    \end{equation*}

    \begin{equation*}\tag{Right distribute}
        \forall \alpha\in \C, v,w\in V \quad \alpha(v+w) = \alpha v + \alpha w
    \end{equation*}
    where we have shortened $s(\lambda,v)$ to $\lambda v$.
\end{definition}

In particular, $(V,+)$ is an abelian group (forgetting the scaling structure).
Some expected properties follow immediately from this definition.

\begin{exercise}
    Prove the following.
    \begin{itemize}
        \item $0 v=\vec{0}$
        \item $v + (-1)v = \vec{0}$
    \end{itemize}
    Note: It is not anywhere in the definition that multiplying $v$ by $-1$ gives the additive inverse of $v$.
    That's what the second part of this exercise is showing.
\end{exercise}

There are many familiar examples.

\begin{example}[Zero]
    $\{ \vec{0} \}$
\end{example}


\begin{example}[$\C$]
    
\end{example}


\begin{example}[Tuples]
    Define $\C^n \coloneq \{ (z_1,z_2,\dots,z_n) \mid z_i \in \C \}$, with addition and scalar multiplication 
    given pointwise:
    \[
        (w_1,\dots,w_n)+(z_1,\dots,z_n) \coloneq (w_1+z+1,\dots,w_n+z_n) \bigand 
        \alpha(z_1,\dots,z_n) \coloneq (\alpha z_1,\dots \alpha z_n).
    \]
\end{example}


\begin{example}[Matrices]
    $M_{m\times n}(\C)$
\end{example}


\begin{example}[Polynomials]
    Define the vector space $\C[x] \coloneq \{ a_n x^n +\cdots+ a_1 x + a_0 \mid a_i \in \C \}$ with addition 
    and scalar multiplication defined pointwise on scalars.
\end{example}


\begin{example}[Functions]
    Define the vector space $\C^{X} \coloneq \{ f:X \to \C\}$, with addition and scalar multiplication defined pointwise:
    \[
        (f+g)(x) \coloneq f(x) + g(x) \quad \text{and} \quad (\alpha \cdot f)(x) \coloneq \alpha \cdot (f(x)).
    \]
\end{example}




Sometimes a vector space sits inside another vector space.

\begin{definition}[Subspace]
    Let $V$ be a vector space, and let $U$ be a subset of $V$.
    Call $U$ a {\bf vector subspace} of $V$ if
    \begin{equation*}\tag{Addition}
        \forall u,v\in V, \quad u+v\in V
    \end{equation*}

    \begin{equation*}\tag{Scalar multiplication}
        \forall u\in U, \lambda\in\C, \quad \lambda u\in U
    \end{equation*}
\end{definition}

\begin{exercise}
    Let $V$ be a vector space, and let $U$ be a subspace of $V$, and suppose $u\in U$.
    Prove that:
    \begin{itemize}
        \item $\vec{0} \in U$
        \item $-u\in U$ 
    \end{itemize}
\end{exercise}

What kinds of functions do we care about for vector spaces?
Well, we have two sorts of structure now, so we want a function to respect both.

\begin{definition}[Linear transformation]
    Let $V$ and $W$ be vector spaces.
    A function $T:V\to W$ is called a {\bf linear transformation} (or {\bf linear map}) if
    \[
        T(u+v) = T(u) + T(w), \quad \text{and} \quad T(\lambda v) = \lambda T(v).
    \]
    If $T:V\to W$ is a linear map, its {\bf kernel} is the set
    \[
        \ker(T) \coloneq \{v\in V \mid T(v)=\vec{0}\}.
    \]
    Its {\bf image} is the set
    \[
        T(V) \coloneq \{ T(v) \mid v\in V\}.
    \]
    Often we will denote the target $T(v)$ of a vector $v\in V$ simply by $Tv$.
\end{definition}

Next, a couple of results that should remind you of group theory.

\begin{proposition}
    Let $V$ and $W$ be vector spaces, and suppose $T:V\to W$ is a linear map.
    Then
    \begin{itemize}
        \item $\ker(T)$ is a subspace of $V$
        \item $T(V)$ is a subspace of $W$
        \item $T(\vec{0}) = \vec{0}$
    \end{itemize}
\end{proposition}

Here are a few quick examples of linear maps between some of the example spaces above.

\begin{example}[Trivial]
    For a vector space $V$, there is always exactly one linear map $\{\vec{0}\} \to V$ and 
    exactly one linear map $V \to \{\vec{0}\}$.
\end{example}

\begin{example}[Embedding]
    Define the map $\C \to \C^n$ by defining
    \[
        \alpha \mapsto \begin{bmatrix}
            \alpha \\
            0 \\ \vdots \\ 0
        \end{bmatrix}
    \]
\end{example}

\begin{example}[Matrix]
    Let $M$ be an $m\times n$ complex matrix.
    View tuples in $\C^k$ as column vectors.
    Define the linear transformation $T_M:\C^n \to \C^m$ by
    \[
        T_M(v) \coloneq Mv.
    \]
\end{example}

In elementary linear algebra courses, linear maps are almost always just given by matrices.
This is possible when we are dealing with {\bf finite-dimensional} vector spaces.
We will mostly have cause only to deal with such vector spaces, but there are many interesting 
examples of infinite-dimensional vector spaces (polynomials, for example).
For more on the correspondence between linear maps and matrices, see 
\href{https://math.libretexts.org/Bookshelves/Linear_Algebra/Book%3A_Linear_Algebra_(Schilling_Nachtergaele_and_Lankham)/06%3A_Linear_Maps/6.06%3A_The_matrix_of_a_linear_map}{this link}.

The reason we're going through all of this is to put ourselves on more sure footing when defining representations,
and the maps between them. 
Let's start by defining some notations that will carry through to our study of categories more generally.

\begin{definition}
    Let $V$ and $W$ be two vector spaces.
    The set of linear maps between $V$ and $W$ is 
    \[
         \Hom_{\Vecc}(V\to W) \coloneq \{ T:V\to W \mid \text{$T$ linear} \}.
    \]
    We also define {\bf endomorphisms} of vector spaces to be (not necesarrily invertible!)
    self-maps:
    \[
        \End_{\Vecc}(V) \coloneq \Hom_{\Vecc}(V\to V).
    \]
    And finally, the invertible self-maps, {\bf automorphisms}:
    \[
        \GL(V) \coloneq \{ T\in\End_{\Vecc}(V) \mid \text{$T$ invertible} \}.
    \]
\end{definition}

Note that, since invertible linear maps are bijective, they are permutations.
Thus $\GL(V)$ is a very special subset of $S_V$, the symmetric group on $V$.
In fact, it's a sub{\bf group}.

\begin{proposition}
    Let $V$ be any vector space.
    The collection $\GL(V)$ is a group.
\end{proposition}

The proof is very similar to proving that $\Aut(G)$ is a group for any group $G$.

Now we're in a position to ask whether, given a group $G$ and a vector space $V$, there might
be a homomorphism $G\to \GL(V)$. This is exactly what a representation is.

\begin{definition}[Representation]
    Let $G$ be a group and $V$ a vector space.
    A {\bf representation} of $G$ on $V$ is a homomorphism 
    \[
        \rho:G\to\GL(V).
    \]
\end{definition}


