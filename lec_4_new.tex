\section*{date TBD}

\red{UNSTABLE}

I want to take a step back and examine some parts of the definition of a representation.
We defined a (complex, linear) representation as a homomorphism $G\to \GL_n(\C)$.
I did this to be a bit more concrete and avoid discussing vector spaces in their own right.
But I now realize that was a tactical mistake, and we should discuss vector spaces.

\begin{definition}
    A (complex) {\bf vector space} is a tuple $(V, s)$ where $V$ is a set, 
    and $s:\C\to V$ is a function,
    satisfying the following properties:
    \begin{equation*}\tag{Addition}
        \forall u,v\in V, \quad u+v\in V
    \end{equation*}

    \begin{equation*}\tag{Zero}
        \exists 0\in V; \forall v\in V \quad 0+v = v+0 = v
    \end{equation*}

    \begin{equation*}\tag{Negative}
        \forall u\in V, \exists v\in V; \quad u+v = v+u = 0
    \end{equation*}

    \begin{equation*}\tag{Left distribute}
        \forall u,v\in V, \quad u+v\in V
    \end{equation*}

    \begin{equation*}\tag{Right distribute}
        \forall u,v\in V, \quad u+v\in V
    \end{equation*}
    where we have shortened $s(\lambda,v)$ to $\lambda v$.
\end{definition}

Some expected properties follow immediately from this definition.

\begin{exercise}
    Prove the following.
    \begin{itemize}
        \item $v + (-1)v = 0$
    \end{itemize}
\end{exercise}

There are many familiar examples.

\begin{example}[Column vectors]
    
\end{example}


\begin{example}[Polynomials]
    
\end{example}